\subsection{Резултати}

В настоящата дипломна работа е разработена система за 3D визуализация на движещи се и стационарни обекти, чрез използването на ултразвук за измерване на дистанцията между стационарни обекти наречени предаватели на сигнал и потенциално движещи се обекти наречени получатели на сигнал. Системата е тествана в реални условия чрез използването на 4 стационарни ултразвукови предавателя и един движещ се приемник. Разработени са възможни сценарии за използване на системата в индустрията и извън нея. Системата е сравнена с вече съществуващи на пазара системи и са изкарани изводи, които сравняват качествата и недостатъците им.

Изследват се възможностите за визуализация на системи с много движещи се и стационарни обекти и ограниченията, които биват наложени от разработената технология спрямо броя на обектите, които биват изобразявани.

Системата е разработена с най-новите практики в софтуерното инженерство, които позволяват бързо и лесно да се модифицират различни модули от приложението без това да създава проблеми в останалите модули и е достъпна да бъде използвана с цел допълнително развитие.

\subsection{Перспективи за развитие}

\subsection{Заключение}

\subsection{Приложение}

\subsection{Използвани източници}
[1] Mathias Pelka* Position Calculation with Least Squares based on Distance Measurements