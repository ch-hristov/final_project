\subsection{Резултати}

В настоящата дипломна работа е разработена система за 3D визуализация на движещи се и стационарни обекти, чрез използването на ултразвук за измерване на дистанцията между стационарни обекти наречени предаватели на сигнал и потенциално движещи се обекти наречени получатели на сигнал. Системата е тествана в реални условия чрез използването на 4 стационарни ултразвукови предавателя и един движещ се приемник. Разработени са възможни сценарии за използване на системата в индустрията и извън нея. Системата е сравнена с вече съществуващи на пазара системи и са изкарани изводи, които сравняват качествата и недостатъците им.

Изследват се възможностите за визуализация на системи с много движещи се и стационарни обекти и ограниченията, които биват наложени от разработената технология спрямо броя на обектите, които биват изобразявани.

Системата е разработена с най-новите практики в софтуерното инженерство, които позволяват бързо и лесно да се модифицират различни модули от приложението без това да създава проблеми в останалите модули и е достъпна да бъде използвана с цел допълнително развитие.

\subsection{Перспективи за развитие}
Разработената система може да бъде развита с цел комерсиална употреба като пълноправен софтуерен продукт, както и да бъде използвана за персонални цели. Подобна система се използва в \cite{vr} за целите на ориентация във увелиячената реалност. Софтуерната имплементация може да бъде разширена с допълнителни модули за анализ на данните. Текущата имплементация приема, че данните получени от ултразвук трансмитери и получатели са валидни. Имплементацията може да бъде разширена като се добавят филтри за смущение и невалидна информация. В текущата статия скоростта на движението на ултразвука във въздуха се счита за константа, но в реални условия тя е пряко зависима от температурата на стаята \cite{vr}, поради тази причина разработката на модул за тест на хардуерната способност да измерва качествено разстоянието между обектите може да бъде разработен. Алгоритъмът, който изчислява позицията на движещите се обекти, използва система от линейни уравнения. Този метод не е оптимален, когато се работи с разстояния, които са приблизителни. Чрез разшириение на Graph Coordinator модула може да бъде добавен нов начин да се определят позициите (Kalman Filter, нелинейни най-малки квадрати и др.). В даден бъдещ момент може да бъде метод за корекция на определените координати според предишни измервания (осредняване на последните \textit{К} измервания и др.).

\subsection{Заключение}
Разработена е система за 3D визуализация на компонентите на система, която използва утразвук за измерване на разстоянията между компонентите на системата. Сравнени са предимствата и недостатъците на съществуващи системи използвани в индустрията. Изследвани са начини за определяне на координатите на компонентите във виртуалното пространство. Имплементирана е софтуерна програма, която извършва визуализацията на компонентите и са изследвани достъпните инструменти, които имат потенциала да помагат в разработката на подобен продукт.

\subsection{Приложение}

\begin{itemize}
    \item Образуване на входящ граф от дистанции
    \item Образуване на работен граф
    \item Определяне на координатите на компонентите на графа
    \item Изобразяване на резултата
\end{itemize}

\subsection{Терминологичен речник}
Дефинираме следните понятия в контекста на този документ:
\begin{enumerate}
    \item Получател - хардуерно устройство, което получава ултразвуков сигнал.
    \item Трансмитер - хардуерно устройство, което изпраща ултразвуков сигнал.
    \item Виртуални координати - координати, които съществуват в пространстовото, в което изобразяваме обектите.
    \item Предавател - думата се използва със същото значение като трансмитер.
    \item Трилатерация - клас от проблеми, които могат да бъдат решени само чрез наличност на измервания на разстояния между обекти \cite{murphy}
    \item 3D - термина се използва със смисъл на координатна система с 3 перпендикулярни един на друг вектора
\end{enumerate}