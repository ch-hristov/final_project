\subsection{Резултати}

В настоящата дипломна работа е разработена система за 3D визуализация на движещи се и стационарни обекти, чрез използването на ултразвук за измерване на дистанцията между стационарни обекти наречени предаватели на сигнал и потенциално движещи се обекти наречени получатели на сигнал. Системата е тествана в реални условия чрез използването на 4 стационарни ултразвукови предавателя и един движещ се приемник. Разработени са възможни сценарии за използване на системата в индустрията и извън нея. Системата е сравнена с вече съществуващи на пазара системи и са изкарани изводи, които сравняват качествата и недостатъците им.

Изследват се възможностите за визуализация на системи с много движещи се и стационарни обекти и ограниченията, които биват наложени от разработената технология спрямо броя на обектите, които биват изобразявани.

Системата е разработена с най-новите практики в софтуерното инженерство, които позволяват бързо и лесно да се модифицират различни модули от приложението без това да създава проблеми в останалите модули и е достъпна да бъде използвана с цел допълнително развитие.

\subsection{Перспективи за развитие}
Разработената система може да бъде развита с цел комерсиална употреба като пълноправен софтуерен продукт, както и да бъде използвана за персонални цели. Подобна система се използва в \cite{vr} за целите на ориентация във увелиячената реалност. Софтуерната имплементация може да бъде разширена с допълнителни модули за анализ на данните. Текущата имплементация приема, че данните получени от ултразвук трансмитери и получатели са валидни. Имплементацията може да бъде разширена като се добавят филтри за смущение и невалидна информация. В текущата статия скоростта на движението на ултразвука във въздуха се счита за константа, но в реални условия тя е пряко зависима от температурата на стаята \cite{vr}.

\subsection{Заключение}
Разработена е система за 3D визуализация на компонентите на система, която използва утразвук за измерване на разстоянията между компонентите на системата. Сравнени са предимствата и недостатъците на съществуващи системи използвани в индустрията. Изследвани са начини за определяне на координатите на компонентите във виртуалното пространство. Имплементирана е софтуерна програма, която извършва визуализацията на компонентите и са изследвани достъпните инструменти, които имат потенциала да помагат в разработката на подобен продукт.

\subsection{Приложение}
