



\subsection{Използвани технологии}
\begin{enumerate}
 \itemЗа разработката на приложението е използвана интегрираната среда за разработка на софтуер \textit{Visual Studio 2017}. 
 \itemИзползваният програмен език е \textit{C Sharp}. 
 \itemЗа визуализацията на примитиви в 3D е използвана библиотеката с отворен код \textit{Helix Toolkit 3D}. Позволява достъп до вече настроен viewport, както и някои базови за работата с 3D обекти функции – ротация, транслация и скалиране.
 \itemЗа олеснена работа със структурата от данни граф е използвана библиотеката с отворен код \textit{QuickGraph}. \textit{BidirectionalGraph<V,E>} е конкретната имплементация на граф използвана в проекта.  Тя представлява неориентиран граф, Използва се, за да се репрезентират обектите като върхове, а разстоянията измерени между обектите като дъги между върховете. Два обекта са свързани с не ориентирана дъга с тегло равно на разстоянието между двата обекта, ако няма измерение на разстоянието между обектите то тогава дъга между двата върха няма.
 \itemЗа тестване на програмата в реални условия са използвани 4 бр. ултразвуков трансмитери и 1 бр. ултразвук получател, с производител : Hexamite. Устройствата са предоставени от др. Димитър Минчев
\end{enumerate}



