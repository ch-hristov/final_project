\subsection{Използвани технологии}
\begin{enumerate}
 \itemЗа разработката на приложението е използвана интегрираната среда за разработка на софтуер \textit{Visual Studio 2017}. 
 \itemИзползваният програмен език е \textit{C Sharp}. 
 \itemЗа визуализацията на примитиви в 3D е използвана библиотеката с отворен код \textit{Helix Toolkit 3D}. Позволява достъп до вече настроен viewport, както и някои базови за работата с 3D обекти функции – ротация, транслация и скалиране.
 \itemЗа олеснена работа със структурата от данни граф е използвана библиотеката с отворен код \textit{QuickGraph}. \textit{BidirectionalGraph<V,E>} е конкретната имплементация на граф използвана в проекта.  Тя представлява неориентиран граф, Използва се, за да се репрезентират обектите като върхове, а разстоянията измерени между обектите като дъги между върховете. Два обекта са свързани с не ориентирана дъга с тегло равно на разстоянието между двата обекта, ако няма измерение на разстоянието между обектите то тогава дъга между двата върха няма.
 \itemЗа тестване на програмата в реални условия са използвани 4 бр. ултразвуков трансмитери и 1 бр. ултразвук получател, с производител : Hexamite. Устройствата са предоставени от др. Димитър Минчев
\end{enumerate}

\subsection{Обосновка за използвани технологии}

\begin{enumerate}
    \item Visual Studio предоставя мощна интегрирана среда за писане на код, компилиране, изпълнение, дебъгване (както за високо така и за машинно ниво), тестване на приложения, дизайн на потребителски интерфейс (форми, диалози, уеб страници, визуални контроли и други), моделиране на данни, моделиране на класове, изпълнение на тестове, пакетиране на приложения и стотици други функции. Могат да се добавят и плъгини, които повишават функционалността на почти всяко ниво – включително добавянето на поддръжка за source-control системи (като Subversion и Visual SourceSafe), добавяне на нови инструменти като редактори и визуални дизайнери за domain-specific languages или инструменти за други аспекти (като например: Team Foundation Server, Team Explorer). \cite{vs}
    
    \item C Sharp (C Sharp, произнася се Си Шарп) е обектно-ориентиран език за програмиране, разработен от Microsoft, като част от софтуерната платформа .NET. Стремежът още при създаването на C# езика е бил да се създаде един прост, модерен, обектно-ориентиран език с общо предназначение. Основа за C Sharp са C++, Java и донякъде езици като Delphi, VB.NET и C. Той е проектиран да балансира мощност (C++) с възможност за бързо разработване (Visual Basic и Java). Те представляват съвкупност от дефиниции на класове, които съдържат в себе си методи, а в методите е разположена програмната логика – инструкциите, които компютърът изпълнява. Програмите на C# представляват един или няколко файла с разширение .cs., в които се съдържат дефиниции на класове и други типове. Тези файлове се компилират от компилатора на C Sharp (csc) до изпълним код и в резултат се получават асемблита – файлове със същото име, но с различно разширение (.exe или .dll). \cite{csharp}
    
    \item Helix Toolkit builds on the 3-D functionality in Windows Presentation Foundation (WPF). While the functionality of 3-D in WPF may fall short of that needed for complex 3-D gaming engines it does have enough merit to stand on its own. This toolkit provides a higher level API for working with 3-D in WPF, via a collection of controls and helper classes. \cite{helix}
    
    \item QuickGraph предлага стабилни имплементации на структурата от данни граф за платформата за разработка .NET. Това включва насочен/ненасочен граф. Имплементирани са често използвани алгоритми като търсене в дълбочина, търсене в дълбочина , A* търсене и т.н. \cite{quickgraph}

    
\end{enumerate}