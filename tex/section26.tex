\subsection{Калман филтър} \label{kalmanSect}

В източник \cite{kalmanTutorial} се изследва приложимостта на Калман филтър за определяне на дадена стойност при присъствие на несигурност или неточност в работните данни. Определянето на координатите на движещите се обекти в източник \cite{bristolBeacons} се извършва с помощта на Калман филтър. 

Популярността на Калман филтър се базира на следните качества, които той притежава

\begin{enumerate}
    \item Добри практически резултати
    \item Удобен е за обработка на данни в реално време
    \item Лесна за реализация имплементация
    \item Измервателните уравнения не трябва да се инвертират
\end{enumerate}

Считайки Гаусово разпределение на грешката Калман филтърът е оптимален т.е минимизира средноквадратичните грешки на параметрите, които се предсказват. При разпределение различно от Гаусово разпределение нелинейни решения водят до по-добри резултати. \\

Разглеждат се 2 стъпки в рекурсивния вариант на ФК (Филтър на Калман) \cite{kalmanTU}).

\begin{enumerate}
    \item Предиктор - тази стъпка служи за предсказване на вектор на състояние спрямо модела
    \item Коректор - коригиране на предсказаните данни
\end{enumerate}

ФК има широка приложимост в системи с понижена чувствителност към повреди (Fault tolerant control). Предимство на ФК е, че изисква ниско количество памет, защото се изисква само предишното състояние на системата.

ФК работи чрез използването на така наречената covariance matrix, симетрична матрица, която, описва корелацията между различните променливи в системата. Симетрична e, защото корелацията, е симетрично свойство.