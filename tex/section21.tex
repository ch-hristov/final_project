\subsection{Метод на най-малките квадрати} \label{squares_algorithm}

Изпозлвайки метода на най-малките квадрати представен в \cite{leastsq} се демонстрира намиране на решение чрез разширение на Теоремата на Питагор за 3D. За да се трансформира горната задача в математически модел е нужно да имаме anchor\cite{leastsq2}, което преставлява ориентировъчна точка в пространстовото, в което ще определяме координатитите. За anchor обект избираме позицията на първия трансмитер във виртуалното пространство. Позицията във виртуалното пространството на трансмитерите е константно и е известно винаги.


\begin{equation} \label{pytEq}
   (x-x_i)^2 + (y-y_i)^2 + (z-z_i)^2=d_i^2
\end{equation}

Уравнение \ref{pytEq} описва взаимовръзката между разстоянието и координатите на два обекта. За да генерализираме уравнението, с индекс \textit{l} означаваме стойностите за anchor обекта. Чрез трансформация на уравнението се достига до следния запис:

\begin{equation} \label{pytEqTransformed}
  2 x (x_i - x_l) - 2 y (y_i - y_l) - 2  z  (z_i - z_l) = d^2_i - d^2_l - k_l + k_i
\end{equation}

За краткост в уравнение \ref{pytEqTransformed} променливата \textit{k} е означена като: 
\begin{equation} \label{kdesc}
    k_i= x^2_i + y^2_i + z^2_i
\end{equation}

Използваме това уравнение за всички получатели, което за брой на получателите = 3 е следното матрично уравнение:

\centerline \\
    2 {\begin{bmatrix}
        (x_2-x_1) & (y_2 - y_1) & (z_2 - z_1)\\
        (x_3-x_1) & (y_3 - y_1) & (z_3 - z_1)\\
        (x_4-x_1) & (y_4 - y_1) & (z_4 - z_1)
    \end{bmatrix}
    \begin{bmatrix}
        x\\y\\z
    \end{bmatrix}
    =
    \begin{bmatrix}
    d^2_1 - d^2_2 - k_1 + k_2\\
    d^2_1 - d^2_3 - k_1 + k_3\\
    d^2_1 - d^2_4 - k_1 + k_4\\
    \end{bmatrix}\\
}

Използвайки линейна алгебра се решава горе посоченото уравнение, чиито резултат са търсените координати.