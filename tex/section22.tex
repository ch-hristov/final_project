\subsection{Метод за следене използвайки ултразвук за увеличена реалност} \label{sectionVr}

В документ \cite{vr} се описва система изградена от 4 получателя, които са позиционирани в ъглите на региона. Описани са 3 равнини [фиг. \ref{fig:planes}], които се използват за изчислението на координатите на точките. Това е еквивалентно на линеризиране на уравненията, за което в \cite{murphy} е споменато като метод, който не дава оптимални резултати в реални условия.  Изчисленията на координатите се извършват с помощта на  Питагоровата теорема. За целта се използват  следните уравнения. \\

\centerline{
    \begin{equation}
        k=i+2
    \end{equation}
}

\centerline{
    \begin{equation} \label{22eq}
        \begin{bmatrix}
                $(x_i - x)^2 + y^2 = d_i_2$\\
                $(x_k - x)^2 + y^2 = d^2_k$
        \end{bmatrix}
    \end{equation}
}

\begin{figure}
    \centering
    \includegraphics{planes}
    \caption{Равнините използвани за пресмятането на координатите}
    \label{fig:planes}
\end{figure}

В документа се разглеждат метод за пресмятане на разстояние между трансмитер и получател. Представено е уравнение за разстояние [уравнение \ref{disteq}].

\centerline{\begin{equation} \label{disteq}
    d = t_b v + d_n
\end{equation}}

където 
$d$ - разстоянието между трансмитер и получател\\
$t_b$ - времето за което звука пътува в пространството\\
$v$ - скоростта на звука в конкретна температура (m/s)\\
$d_n$ - минималното разстояние, което получателя може да отработи\\

Скоростта на звука е пряко свързана с температурата, в която пътува той. Изследвания, които изискват силна точност в измерените разстояния е редно да вземат предвид температурата \cite{speedOfSounds}.