\subsection{Увод}
\tab През изминалите години се изследват системи, които използват радио сигнали за определяне на позицията в пространстовото на обекти. Такъв вид системи изискват по-малко инфраструктура, но предлагат по-неточни резултати в сравнение с останалите видове системи \cite{CarlosMedina}. Грешката в точността на технологиите WiFi \cite{wifi}, ZigBee \cite{zigbee} и RFID (Radio Frequency IDentification) \cite{rfid} се измерва до няколко метра. В сравнение с тези системи ултразвукът има точност до няколко сантиметра \cite{CarlosMedina} \cite{columbia}. Настоящата дипломна работа се фокусира върху разработване на платформа за тримерна визуализация на обекти в ограничен регион, използвайки измервателните възможности на ултразвук трансмитери и получатели. Използването на ултразвук за определянето на позицията на обекти в пространството, има потенциал да бъде употребявана в региони, в които GPS сигналът е прекалено слаб или заглушен \cite{yonei}. Разработката включва изследване на възможностите, дефиниране на задачи, анализиране на задачите, сравнение на възможните подходи, решение и анализ на резултатите. 

\subsection{Цел}
\tabОсновната цел на настоящата дипломна работа е да бъдат изследвани наличните алгоритми за определяне на координатите на обекти, които могат да се движат, в 3 измерения и да бъде разработена платформа за тримерна визуализация, базирана на ултразвукова система за позициониране, която да изобразява следените обекти с максимална грешка до 5 фута. Като изискване е счетено високото качеството на програмния код с цел, да бъде лесно използваемо за индустриални и академични цели. За минимален брой обекти, за които системата трябва да може коректно да изобразява движението, са счетени - 2 движещи се обекта и 4 статични обекта, които са недвижеши се.

\subsection{Дефиниции}
Дефинираме следните понятия в контекста на този документ:
\begin{enumerate}
    \item Получател - хардуерно устройство, което получава ултразвуков сигнал.
    \item Трансмитер - хардуерно устройство, което изпраща ултразвуков сигнал.
    \item Виртуални координати - координати, които съществуват в пространстовото, в което изобразяваме обектите.
    \item Предавател - думата се използва със същото значение като трансмитер.
    \item Трилатерация - клас от проблеми, които могат да бъдат решени само чрез наличност на измервания на разстояния между обекти \cite{murphy}
    \item 3D - термина се използва със смисъл на координатна система с 3 перпендикулярни един на друг вектора \cite{3d}
\end{enumerate}

\subsection{Задачи}
За постигане на поставената цел е необходимо да бъдат изпълнени следните задачи
\begin{itemize}
    \item Да се изследва приложимостта на използване на ултразвуков сигнал за построяването на 3D картина в реално време и приложимите решения за този проблем.
      
    \item Да се анализират възможните методи за определяне на координатите в 3D виртуално пространство чрез използването на трансмитери и получатели, за да се определи, дали такава система би била удачна за нуждите на индустрията.
    
    \item Да се анализира сложността на работата на системата и качеството на визуализацията на обектите.
    
    \item Да се имплементира архитектурното решение разработено в този документ - Глава 
    
    \item Да се имплементира алгоритъм за определяне на координатите на обектите от измерения на разстояние.
    
    \item Да се имплементира начин за контрол на броя на статичните и движещите се обекти в системата под формата на меню за опции.
    
    \item Да се разработи софтуерна програма за визуализация в 3D на определените координати, която използва разработения алгоритъм, за да извършва координацията на обектите в 3D виртуално пространство.
    
    \item Да се тества работата на софтуерната програма в реални условия.
    
    \item Да се тестват ограниченията на програмата спрямо броя на движещи получатели и стационарни трансмитери.
\end{itemize}