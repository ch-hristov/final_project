\subsection{Увод}
\tab Ултразвукът е звук (>=20 kHz \cite{batmobile}) с малка дължина на вълната, който предлага възможности за бърз пренос на данни. През изминалите години се изследват системи, които използват радио сигнали за определяне на позицията в пространстовото на обекти. Такъв вид системи изискват по-малко инфраструктура, но предлагат по-неточни резултати в сравнение с останалите видове системи \cite{CarlosMedina}. Грешката в точността на технологиите WiFi \cite{wifi}, ZigBee \cite{zigbee} и RFID (Radio Frequency IDentification) \cite{rfid} се измерва до няколко метра. В сравнение с тези системи ултразвукът има точност до няколко сантиметра \cite{CarlosMedina} \cite{columbia}. Настоящата дипломна работа се фокусира върху разработване на платформа за тримерна визуализация на обекти в ограничен регион, използвайки измервателните възможности на ултразвук трансмитери и получатели. Използването на ултразвук за определянето на позицията на обекти в пространството, има потенциал да бъде употребявана в региони, в които GPS сигналът е прекалено слаб или заглушен \cite{yonei}. Определянето на координати в затворени пространства е проблем, който технологии като GPS не могат да адресират достатъчно точно \cite{gpsIsBadIndoor}, тъй като тяхната точност намалява драстично в такъв вид обстановка. Нарастването на дяла на сфери като IoT (Internet of Things), VR (Virtual reality) и AR(Augmented reality) изискват достъпен и точен начин за проследяване на движещи се обекти в затворени пространства. Предимство на този вид системи е, че не се изисква физическа връзка между различните компоненти на системата. Ултразвуковите системи за позициониране имат потенциал за комерсиална употреба, както е демонстрирано от документ \cite{batmobile}, в който се описва приложимостта на следене чрез ултразвук при при навигация на клиенти в супермаркет. Информацията, която може да бъде извлечена от координатите и контекста на системата, може да бъде използвана за множество цели - някои, от които са - следене на позицията в пространството за предоставяне на специализирани реклами според позицията в даден регион, видео игри, медицина и индустриални цели \cite{bristolBeacons}. Съществуват опасения за извличането на лична информация, която може да бъде обработване под формата на 'metadata' - данни, които идват от контекста на употреба. Текущата разработка включва изследване на възможностите, дефиниране на задачи, анализиране на задачите, сравнение на възможните подходи и решения и анализ на резултатите на ултразвукови системи за позициониране. Анализирани са качествата и недостатъците на този вид системи и се разглеждат възможностите за бъдещото развитие на предложената система. Разгледани са технологии, които са удачни за решението на задачата и са представени техните качества. Разгледана е архитектурата и модулите, които са разработени като софтуерно решение на задачата. 

\subsection{Цел}
\tabОсновната цел на настоящата дипломна работа е да бъдат изследвани наличните технологии за определяне на координатите на обекти, които могат да се движат в 3 измерения и да бъде разработена софтуерна програма за тримерна визуализация на обекти в пространтвото, базирана на ултразвукова система за позициониране, която да изобразява статичните и движещите се обекти участващи в системата. Като изискване е счетено високото качеството на програмния код, с цел, платформата да бъде лесно приспособима за бъдещи индустриални и академични цели. За минимален брой обекти, за които системата трябва да може коректно да изобразява движението, са счетени - 1 движещ се обект и 4 статично позиционирани в пространството обекта. 

\subsection{Задачи}
За постигане на поставената цел е необходимо да бъдат изпълнени следните задачи.

\begin{itemize}
    \item Да се изследва приложимостта на използване на ултразвуков сигнал за построяването на 3D картина в реално време и приложимите решения за този проблем.
      
    \item Да се анализират възможните методи за определяне на координатите в 3D виртуално пространство чрез използването на трансмитери и получатели, за да се определи, дали такава система би била удачна за нуждите на индустрията.
    
    \item Да се анализира сложността на работата на системата и качеството на визуализацията на обектите.
    
    \item Да се създаде софтуерна архитекура, която да бъде удачна за бъдещо развитие. 
    
    \item Да се имплементира алгоритъм за определяне на координатите на обектите от измерения на разстояние.
    
    \item Да се имплементира начин за контрол на броя на статичните и движещите се обекти в системата под формата на меню за опции.
    
    \item Да се разработи софтуерна програма за визуализация в 3D на определените координати, която използва разработения алгоритъм, за да извършва координацията на обектите в 3D виртуално пространство.
    
    \item Да се тества работата на софтуерната програма в реални условия.
    
    \item Да се тестват ограниченията на програмата спрямо броя на движещи получатели и стационарни трансмитери.
\end{itemize}