\subsection{Увод}
Настоящата дипломна работа се фокусира върху разработване на платформа за тримерна визуализация базирана на ултразвукова система за позициониране. Използвани са най-съвременните технологии, както следва: Операционна система Microsoft Windows 10, платформа Microsoft .NET Framework, интегрирана среда за разработка Visual Studio 2017 Community Edition, език за програмиране C#. Софтуерна програма, която използва ултразвук за определянето на позицията на обекти в пространството, има потенциал да бъде употребявана в региони, в които GPS сигналът е прекалено слаб или заглушен [1].

\subsection{Цел}
Основната цел на настоящата дипломна работа е да се разработи платформа за тримерна визуализация, базирана на ултразвукова система за позициониране.

\subsection{Задачи}
За постигане на поставената цел е необходимо да бъдат изпълнени следните задачи:\\\\
•	Да се изследва:\\
\tabВъзможността на използване на ултразвуков сигнал за построяването на 3D картина в реално време и приложимите решения за този проблем.\\\\
•	Да се анализират :\\
\tab1.	Възможните ултразвукови предаватели и приемници, за да се определи, дали такава система би била удачна за нуждите на индустрията.\\
\tab2.	Сложността на работата на системата и качеството на визуализацията на обектите.\\\\
•	Да се разработят:\\
\tab1.	Алгоритъм за определяне на координатите на обектите от измеренията за разстояние, които приемниците и предавателите измерват.\\
\tab2.	Софтуерна програма за визуализация в 3D на определените координати.\\\\
•	Да се тества:\\
\tab1.	Работата на софтуерната програма в реални условия.\\
\tab2.	Ограниченията на програмата спрямо броя на движещи приемници и стационарни предаватели.\\\\