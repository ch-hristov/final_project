\subsection{Увод}
\tabНастоящата дипломна работа се фокусира върху разработване на платформа за тримерна визуализация на обекти в ограничен регион, използвайки измервателните възможности на ултразвук трансмитери и получатели. Използването на ултразвук за определянето на позицията на обекти в пространството, има потенциал да бъде употребявана в региони, в които GPS сигналът е прекалено слаб или заглушен [1].

\subsection{Цел}
\tabОсновната цел на настоящата дипломна работа е да се разработи платформа за тримерна визуализация, базирана на ултразвукова система за позициониране.

\subsection{Дефиниции}
Дефинираме следните понятия в контекста на този документ:
\begin{enumerate}
    \item Получател - хардуерно устройство, което получава ултразвуков сигнал.
    \item Трансмитер - хардуерно устройство, което изпраща ултразвуков сигнал.
    \item Виртуални координати - координати, които съществуват в пространстовото, в което изобразяваме обектите.
    \item Предавател - думата се използва със същото значение като трансмитер.
    \item Трилатерация - клас от проблеми, които могат да бъдат решени само с измерване на разстояния \cite{murphy}
\end{enumerate}

\subsection{Задачи}
За постигане на поставената цел е необходимо да бъдат изпълнени следните задачи
\begin{itemize}
    \item Да се изследва:
     на използване на ултразвуков сигнал за построяването на 3D картина в реално време и приложимите решения за този проблем.
    \itemДа се анализират :
    \begin{enumerate}
     \itemВъзможните ултразвукови трансмитери и получатели, за да се определи, дали такава система би била удачна за нуждите на индустрията.
     \itemСложността на работата на системата и качеството на визуализацията на обектите.
    \end{enumerate}
    \itemДа се разработят:
    \begin{enumerate}
    \itemАлгоритъм за определяне на координатите на обектите от измерения на разстояние
    \itemСофтуерна програма за визуализация в 3D на определените координати.
    \end{enumerate}
    \itemДа се тества:
    \begin{enumerate}
    \itemРаботата на софтуерната програма в реални условия.
    Ограниченията на програмата спрямо броя на движещи получатели и стационарни трансмитери.
    \end{enumerate}
\end{itemize}