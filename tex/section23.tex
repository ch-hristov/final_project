\subsection{Определяне на позиция в 3 измерения с помоща на трилатерация и приблизителни разстояния}

В документ \cite{murphy} е разработена система за следене на товарите в 3 измерения в мина чрез използване на трансмитери и получатели [фиг. \ref{fig:mine}]. Използвани са радио трансмитери. Разработен е специализиран лазер, който изчислява височината на товара, без който проблема за 3 измерения се е считал за нерешим. Проблем представляват разликите във височината и останалите.

\begin{figure}
    \centering
    \centerline{\includegraphics{dt1}}
    \caption{Разположение на получателите и предавателя в мина}
    \label{fig:mine}
\end{figure}

\begin{equation}\label{circleEq}
    (x-x_i)^2 + (y-y_i)^2 +(z-z_i)^2=r_i^2
\end{equation}

Разглежда се възможността за съствяне система от \textit{N} на брой нелинейни уравнения използвайки формулата за сфера (уравнение \ref{circleEq}). Задачата се моделира като търсене на пресечни точки на N на брой сфери за всеки трансмитер.

Решението на гореспоменатата нелинейна система от уравнения \emph{се счита за неизползваема} тъй като полученото уравнение е нелинейно и е от висока степен. Когато разстоянията, които са измерени са точни, а не са приблизителни, линезирането на системата от уравнения е удачно. В този случай задачата се преобразува в търсенето на пресечната точка на няколко равнини.  За долна граница на бройката на приемници се прима 4 бр. Системи с по-малко от тази бройка приемници се считат за неизползваеми. \\

\emph{
    Презентиран е начин да се линеризират уравненията, но той няма да бъде разгледан в настоящия документ, тъй като в реални условия системите за измерване имат приблизителни стойности, което води до незадоволителни резултати от линеризираното уравнение.
}
\\

Разглеждат се 3 метода за работа с приблизителни разстояния

\begin{itemize}
    \item \strong{Линейни най-малки квадрати} \\ Този метод предоставя по-задоволителни резултати в сравнение с решаване на проблема чрез решение на задачата с линеризиране на уравненията.  Този метод, не дава задоволителни резултати, защото резултатите определят координатите с голяма грешка (повече от 5 фута). Тъй като разстоянията са приблизителни се решава уравнение \ref{matrixeq}
    
    \begin{equation} \label{matrixeq}
      A \vec{x} \approx \vec{b} 
    \end{equation}
    
\end{itemize}

    
    