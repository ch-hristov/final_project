\subsection{Обобщение}

Трилатерацията е ефективен метод, когато в дадената задача се работи само с малки грешки в измерванията на разстоянията \cite{trilat}, защото е намерена корелация между измереното разстояние и големината на грешката при измерване. Грешните измервания се наблюдават най-често при измервания, когато измерените стойности са под голям ъгъл или на голямо разстояние. Итеративни методи като линейни най-малки квадрати, които позволяват да се намери решение с приблизителни разстояния са ефективни когато измерените стойности съдържат малка грешка,а при по-голяма грешка е редно да се използват други методи като нелинейни най-малки квадрати. Грешките често са причинени от препятствия и други фактори \cite{yonei}. Съществуват различни методи за обработване на грешките, един от които е филтър за грешка при разстоянията, който се разглежда в \cite{yonei} наречен Калман филтър. Когато грешката при измерване е по-голяма е редно да се използват по-сложни за имплементация методи като нелинейни най-малки квадрати, които водят до по-добри резултати. Използването на такъв вид пресмятания улеснява създаването на системи, които работят в реално време тъй като елиминира нуждата за пресмятане на ъгли. Трилатерацията е метод за пресмятане на координати, който може да бъде приложен в други сфери извън конкретните проблеми и решения демонстрирани в този документ. \\В глава \ref{yoneiSection} се дискриминират 2 вида ултразвукови системи за позициониране ( активни и пасивни ). Разработената ултразвукова система за позициониране в тази дипломна работа е основана на данните от пасивна система. В секция \ref{sectionVr} се разглежда зависимостта между температурата и влажността на околната среда и ефектите върху качеството на измерването. В секция \ref{kalmanSect} се разглежда същността на Калман филтър за данни, който може да бъде използван за подобряване на резултати от измерванията получени от сензори. 