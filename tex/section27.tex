\subsection{Обобщение}

Трилатерацията е ефективен метод, когато в дадената задача се работи само с малки грешки в измерванията на разстоянията \cite{trilat}, защото е намерена корелация между измереното разстояниe и големината на грешката при измерване. Грешните измервания се наблюдават най-често при измервания, когато измерените стойности са под голям ъгъл или на голямо разстояние. Итеративни методи като линейни най-малки квардрати, които позволяват да се намери решение с приблизителни разстояния са ефективни когато измерените стойности съдъръжат малка грешка. Тази грешка може да бъде причинена от препядствия и други фактори \cite{yonei}. Съществуват различни методи за обработване на грешките, един от които е филтър за грешка при разстоянията, който се разглежда в \cite{yonei}. Когато грешката при измерване е по-голяма е редно да се използват по-сложни за имплементиране методи като нелинейни най-малки квадрати, които водят до по-добри резултати. Използването на такъв вид пресмятания услеснява създаването на системи, които работят в реално време тъй като елиминира нуждата за пресмятане на ъгли. Трилатерацията е метод за пресмятане на координати, който може да бъде приложен в други сфери извън конкретните проблеми и решения демонстрирани в този документ. В глава 2.4 се дискриминират 2 вида ултразвукови системи за позициониране ( активни и пасивни ).